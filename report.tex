\documentclass[a4paper,12pt,titlepage]{article}
\usepackage[a4paper,backref]{hyperref}
%\usepackage[utf8]{inputenc}
\usepackage[margin=2cm]{geometry}
\usepackage[T1]{fontenc}
\usepackage{moreverb}
\usepackage{fancyhdr}
\usepackage{graphicx}
\usepackage{amssymb, amsmath}
\title{Taisto -- Final report \thanks{Qt Codecamp 2009 -- University of Helsinki}}
\author{Jussi Kokkala \and Mika Laitinen \and Mikko Sysikaski}
\date{\today}
\begin{document}
\maketitle

\section{Introduction}
Taisto is a 2D multiplayer top view shooting game over network. The objective of human players is to ultimately reach level 16. However, the path to gold and glory is bumby and rough and guarded by furious one-eyed monsters, who are not willing to compromise. 

Fortunately to the brave humans, more weapons become available deeper in the dungeons. This plays a significant role in the success of the humans, since monsters are generally known for huge passion for warfare, but not for skilled usage of modern weapons. In other words, \textbf{you} are up to the task of beating an overwhelming number of monsters with a skilled usage of overwhelming weapons. Good luck, fellow.

\section{Features and limitations}

\subsection{Game features}

The game is run in a server where multiple players can connect.
As an important remark, the server, once started, does not block new players from
joining at any point of the game. Instead, it is always possible to join into
an ongoing game.

The players have multiple weapons to beat the enemies with. Most weapons have limited amount of bullets, and players can gain bullets by collecting bullet packs that spawn to the game area. Also health packs appear in the game, and some enemies drop them when they die.

Players have finite health, and health is lost when players are shot by enemies or other players. Players lose all their bullets and respawn when they are killed.

The game also contains a scoreboard of all players. The scoreboard displays name, kill count and death count of each player.

\subsection {Limitations and known bugs}
Currently, when the game is completed, the player gets only slightly unclear indicator
of this: bullet and health packs emerge to the last base. The game still
continues though, so the statistics still change and people can be killed - but
if the game works as intended, there should not be any new enemies to be killed.

Also a list of currently known bugs:
\begin{itemize}
\item Bullets can fly through walls sometimes.
\item Sometimes enemies can push other enemies through walls.
\item Long nicknames break the statistics view currently.
\end{itemize}


\section{User's guide}

\subsection{Gameplay}
The game commences with humans conquering base 1. Humans have access to five weapons, namely Bead Gun, Shotgun, Machine Gun, Electrogun and Rocket Launcher. However, humans have an unlimited supply of bullets for only one weapon, the Bead Gun. Bullets for every other weapon are available in bullet packs constanly popping to conquered bases. Also health packs pop up in conquered bases.

\subsection{Conquering bases}
In order to advance in the game, the objective of the humans is to conquer new bases. Bases can be conquered by advancing from an previously conquered base to a new one, and eliminating all monsters in the new base and before it. After conquering a base, new health and bullet packs will start to pop up the new base.

\subsection{Losing bases}
It is also possible to lose a conquered base in case more enemies than human players are present in the base. This only applies to the lastly conquered base; therefore it is not possible to lose multiple bases at once.

\subsection{Playing instructions}
The character the player is controlling is always pointing its gun to one direction. This direction is where the player is facing, so if the player is facing to right, and moves forward, the player moves right. The health bar is located in the lower left corner, and the boxes next to it illustrate the selected weapon, its name and the bullets remaining.

Friendly fire is also enabled, so every bullet can harm every character in the game.
\subsection{Controls}
The following buttons control the gameplay:
\begin{itemize}
\item W - move player forward
\item S - move player backward
\item A - move player left
\item D - move player right
\item Right Arrow - turn the gun direction to right (clockwise)
\item Left Arrow - turn the gun direction to left (counter-clockwise)
\item Space - shoot
\item Numbers 1-5 - changing weapons
\item Tab - show player statistics
\item C - open chat dialog
\end{itemize}

\section{How to compile and run}

The game is divided into two programs: Server and Client. Both programs have their own directories in the source code directory and only few source code files are shared among them.

\subsection{Server}

Server can be compiled with the following:
    \begin{verbatimtab}[3]
        cd server
        qmake
        make
    \end{verbatimtab}

To run the server, just run the compiled executable server(.exe). The game starts immediately, and the first players should join as soon as possible to maximize the gameplay experience.

\subsection{Client}

Client can be compiled with the following:
    \begin{verbatimtab}[3]
        cd client
        qmake
        make
    \end{verbatimtab}
If you want to use Qt's own painting system instead of OpenGL, edit file renderarea.h in Client folder and remove the following line and compile again: 
    \begin{verbatimtab}[3]
        #define USE_OPENGL
    \end{verbatimtab}
Do this if your computer doesn't have OpenGL support.\\
To run the client, run compiled executable client(.exe) and then enter the server address into the startup dialog.

\section{Program design}

\subsection{Overal design decisions}

As we only had three days of time to create the game, our main design principle was to implement things in a way that is fast and easy to code, even at the cost of code clearness and maintainability. Based on this principle, we made the following design desicions in creating the game:
\begin{itemize}
\item Both client and server are \textbf{single threaded}. Usin multiple threads would have made the program design more complex, as we would have needed to worry about synchronization and possible hard-to-find bugs. The only area where we thought multiple threads might have been handy was reading input from network, but using a single thread was no issue here either as Qt has good support for non-blocking IO.
\item All networking is done using \textbf{TCP protocol}. UDP would have been a natural choise for an interactive game where all places of players have to be updated many times every second, but using TCP exclusively allowed us to not have to worry about lost packets and packets arriving in different order than they were sent.
\item Game uses \textbf{constant-size physics frame}. This choise was made so that we wouldn't have to think whether physics calculations yield different values if done with different frame times.
\end{itemize}

\subsection {Algorithmic solutions}
Several non-trivial algorithms were implemented during the implementation 
process of Taisto. For example, the hit detection is done by searching the 
distance of the objects from the line the bullet is traversing. This is done
with a standard algorithm for finding the distance of a point from a straight.

The electrogun shooting effect is made with the following algorithm:
\begin{enumerate}
\item Build a minimum spanning tree out of the player himself and the enemies
in range
\item Instead of drawing straight lines between the enemies, some random factor
is generated by splitting each edge of the tree so that the middle point (x,y)
is tracked down, and each x- and y-coordinates get added or substracted at most
N pixels, roughly following a normal distribution so that lesser changes are
more probable. Then we have a new point $(x_0,y_0)$, and now we recursively
call the edge-splitting algorithm for the newly generated two edges, until the
edge to be processed is too short.
\end{enumerate}


\appendix
\section{Cheat codes}
The game has a cheat code to allow testing the guns easily. The cheat is activated by typing the word \emph{ibeatyou} during normal gameplay (not in chat dialog). Doing this gives the player 1000000 bullets to every weapon.

\section{Gameplay images}
\begin{center}
\includegraphics[scale=0.4]{electrogun.png}
\\
Image 1: Electrogun in action, green player is a human player
\end{center}
\begin{center}
\includegraphics[scale=0.4]{rocket.png}
\\
Image 2: Rocket launcher blasting off
\end{center}

\end{document}
