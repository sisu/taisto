\documentclass[a4paper,11pt]{article}
\usepackage[a4paper,backref]{hyperref}
%\usepackage[utf8]{inputenc}
\usepackage[T1]{fontenc}
\usepackage{moreverb}
\usepackage{fancyhdr}
\usepackage{graphicx}
\usepackage{amssymb, amsmath}
\title{Taisto}
\author{Jussi Kokkala, Mika Laitinen, Mikko Sysikaski}
\date{\today}
\begin{document}
\maketitle
\section*{Preview}
Taisto is a 2D multiplayer first person shooter game over network. The objective of human players is to ultimately reach level 16. However, the path to gold and glory is bumby and rough and guarded by furious one-eyed monsters, who are not willing to compromise. 

Fortunately to the brave humans, more weapons become available deeper in the dungeons. This plays a significant role in the success of the humans, since monsters are generally known for huge passion for warfare, but not for skilled usage of modern weapons. In other words, \textbf{you} are up to the task of beating an overwhelming number of monsters with a skilled usage of overwhelming weapons. Good luck, fellow.

\section*{Gameplay}
The game commences with humans conquering base 1. Humans have access to five weapons, namely Bead Gun, Shotgun, Machine Gun, Electrogun and Rocket Launcher. However, humans have an unlimited supply of bullets for only one weaponn, Bead Gun. Bullets for every other weapon are available in bullet packs constanly popping to conquered bases. Also health packs pop up in conquered bases.

\subsection*{Conquering bases}
In order to advance in the game, the objective of the humans is to conquer new bases. Bases can be conquered by advancing from an previously conquered base to a new one, and eliminating all monsters in the new base and before it. After conquering a base, new health and bullet packs will start to pop up the new base.

\subsection*{Losing bases}
It is also possible to lose a conquered base in case an enemy arrives to an abandoned base. This only applies to the lastly conquered base, though.

\section*{Playing instructions}
\subsection*{Controls}


Screenshottej???
\end{document}
